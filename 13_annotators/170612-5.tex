\exercise

Given a topic annotator, like TagME:
\begin{itemize}
  \item detail its goal
  \item define the features link probability of an anchor, commonness of a
    wiki-page w.r.t. an anchor
  \item define the ratio behind the Milne-Witten similarity measure and for
    which purpose it is used in TagME.
\end{itemize}

\solution

The goal of a topic annotator is to identify a sequence of terms in an input
text and to annotate them with un-ambiguous entities drawn from a catalog, for
example augmenting the text with hyperlinks to Wikipedia pages. For the second
point, see the solution of Exercise \ref{13_annotators:tagme}.

The Milne-Witten similarity measure is
%
$$sr(p_a,p_b) = \frac{\log(\max(|A|,|B|)) - \log(|A \cap B|)}{\log(|W|) -
  \log(\min(|A|,|B|))},$$
%
where $p_a$ and $p_b$ are two articles, $A$ and $B$ are the set of articles that
link to $p_a$ and $p_b$ respectively, and $W$ is the set of all articles in
Wikipedia. The formula above is used in TagME to disambiguate the sense of an
anchor $a$, viz. to select the correct page that $a$ must point to among all the
possible pages, i.e. senses of $a$.