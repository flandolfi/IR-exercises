\exercise

Given the sequence $S = (1, 2, 1, 1, 4, 7, 7, 4, 1, 3, 9)$, show:
%
\begin{itemize}
  \item the PForDelta encoding with b=2 and base=0,
  \item the Elias-Fano encoding (warning: remind that EF encodes monotonic
    sequences).
\end{itemize}

\solution

In PForDelta we encode values in the interval $[base,base+2^b-1]$ moving
them to the interval $[0, 2^b-1]$ and storing their representation in $b$ bits.
Values outside this range are marked with an escape sequence
$\triangleright=00$, and encoded in a separate list $L$.

In the given case, since $base=0$, we do not translate the items, but we store
their binary representations
%
$$01\;10\;01\;01\;\triangleright\;\triangleright\;\triangleright\;\triangleright\;01\;11\;\triangleright,$$
%
with the exceptions list
$$L=(4,7,7,4,9).$$

To encode $S$ with Elias-Fano, we create a monotonic sequence decompressing $S$
with gap encoding:
%
$$S'=(1,3,4,5,9,16,23,27,28,31,40).$$
%
Then, we follow the procedure explained in Exercise
\ref{09_posting_compression:elias} on $S'$.
