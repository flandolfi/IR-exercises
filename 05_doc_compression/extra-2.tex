\exercise

Show how to synchronize via rsync the new file $f_{old} = \text{\tt
“abacccdac”}$ (on the server) using the old file $f_{new} = \text{\tt
“cabaddaccc”}$ (on the client) and blocks of size 3 chars. Then show the
execution of the algorithm zsync, using blocks of size 2 chars.

\solution

\begin{description}

  \item[rsync] Client starts computing the hashes of $f_{old}$ divided by blocks
  of 3 chars:
  %
  \begin{align*}
    \underbracket{\vphantom{\tt g}\texttt{a b a}}_{H_1}\
    \underbracket{\vphantom{\tt g}\texttt{c c c}}_{H_2}\
    \underbracket{\vphantom{\tt g}\texttt{d a c}}_{H_3}
  \end{align*}
  %
  The client sends the hashcodes to the server, which compares them to the ones
  produced by the rolling hash on $f_{new}$:
  %
  \begin{align*}
    \texttt{c}\
    \underbracket{\vphantom{\tt g}\texttt{a b a}}_{H_1}\
    \texttt{d}\
    \underbracket{\vphantom{\tt g}\texttt{d a c}}_{H_3}\
    \texttt{c c}
  \end{align*}
  %
  The server then sends: "{\tt c}", $H_1$, "{\tt d}", $H_3$, "{\tt cc}".

  \item[zsync] Server starts computing the hashes of $f_{new}$ divided block by
  block:
  %
  \begin{align*}
    \underbracket{\vphantom{\tt g}\texttt{c a}}_{H_1}\
    \underbracket{\vphantom{\tt g}\texttt{b a}}_{H_2}\
    \underbracket{\vphantom{\tt g}\texttt{d d}}_{H_3}\
    \underbracket{\vphantom{\tt g}\texttt{a c}}_{H_4}\
    \underbracket{\vphantom{\tt g}\texttt{c c}}_{H_5}
  \end{align*}
  %
  The server sends the hashcodes to the client, which compares them to the ones
  produced by the rolling hash on $f_{old}$:
  %
  \begin{align*}
    \texttt{a}\ \
    \mathrlap{\underbracket{\vphantom{\tt g}\phantom{\texttt{b}\ \
    \texttt{a}}}_{H_2}}
    \texttt{b}\ \
    \mathrlap{\overbracket{\vphantom{\tt d}\phantom{\texttt{a}\ \
    \texttt{c}}}^{H_4}}
    \texttt{a}\ \
    \mathrlap{\underbracket{\vphantom{\tt g}\phantom{\texttt{c}\ \
    \texttt{c}}}_{H_5}}
    \texttt{c}\ \
    \mathrlap{\overbracket{\vphantom{\tt d}\phantom{\texttt{c}\ \
    \texttt{c}}}^{H_5}}
    \texttt{c}\ \  \texttt{c}\ \  \texttt{d}\ \
    \underbracket{\vphantom{\tt g}\texttt{a}\ \  \texttt{c}}_{H_4}
  \end{align*}
  %
  Then client sends the vector: 01011. The server then computes the \emph{gzip}
  of the unmatched substrings given the matched ones:
  %
  \begin{table}[H]
    \centering
    \ttfamily
    \begin{tabular}{r|lcl}
    b a a c c \colorbox{pink}{a}c c &\colorbox{yellow}{a d} \$ & &
    $\langle 3, 1, \text{\tt d} \rangle$ \\
    b a a c c a c c & a d \colorbox{yellow}{\$} & &
    $\langle 0, 0, \text{\tt \$} \rangle$ \\
    \end{tabular}
  \end{table}

\end{description}
